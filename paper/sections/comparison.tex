%------------------------------------------------------------------------------
\section{Comparison Methods}
\label{sec:comparison}
%------------------------------------------------------------------------------

\subsection{Double Machine Learning}

Double Machine Learning \citep{chernozhukov2018double} estimates the partially
linear model:
\[
Y = \theta(X) \cdot T + g(X) + \varepsilon
\]

The LinearDML implementation assumes $\theta(X) = X'\beta$, a linear function of
covariates. This specification provides consistent estimation when the
conditional average treatment effect is indeed linear in $X$, but may exhibit
bias when true heterogeneity involves nonlinear patterns such as higher-order
interactions, threshold effects, or periodic components.

\subsection{Causal Forests}

Causal Forests \citep{athey2019generalized} estimate heterogeneous treatment
effects using a forest of honest regression trees. Each tree partitions the
covariate space adaptively, estimating local treatment effects within each leaf.
The method provides confidence intervals based on the infinitesimal jackknife
but requires the effect function to be well-approximated by piecewise constant
functions on rectangular partitions.

\subsection{Quantile Forests}

For comparison of treatment effect distribution estimation, we also consider
Quantile Forests that estimate conditional quantiles of the potential outcomes.
By fitting separate forests for treated and control outcomes, one can estimate
quantiles of the treatment effect distribution, providing a nonparametric
benchmark for our quantile estimates.
